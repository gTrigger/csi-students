\documentclass[a4paper,12pt]{article} %размер бумаги устанавливаем А4, шрифт 12пунктов
%\usepackage[T2A]{fontenc}
\usepackage[utf8]{inputenc}
\usepackage{csquotes}
\usepackage[english,russian]{babel}%используем русский и английский языки с переносами
\usepackage{biblatex}
\bibliography{refs}
\usepackage{amssymb,amsfonts,mathtext,enumerate,float} %подключаем нужные пакеты расширений
\usepackage{textcomp}
\usepackage{adjustbox}
\usepackage{graphicx} 
%\graphicspath{{images/}}%путь к рисункам
\makeatletter
\makeatother
\usepackage{sagetex}
\usepackage{geometry} % Меняем поля страницы
\geometry{left=2cm}% левое поле
\geometry{right=1cm}% правое поле
\geometry{top=1cm}% верхнее поле
\geometry{bottom=2cm}% нижнее поле




%\usepackage{tabularx} 

\usepackage{epstopdf}
\usepackage{amsmath} %отображение математической нотации
\usepackage{caption, subcaption} %подписи}
\usepackage{indentfirst}%отступ вначале параграфа
\usepackage{pscyr}
%\usepackage{natbib}
%\usepackage{ragged2e} %для таблиц
 
\captionsetup[table]{labelsep = endash, singlelinecheck=false}
\captionsetup[figure]{name = Рисунок, labelformat=simple, labelsep = endash}

\begin{document}

\paragraph{Цель работы:}Исследование точностных свойств систем управления.%*-без нумерации
\paragraph{Исходные данные.} В таблице 1 приведены передаточная функция ОУ, характеристики задающих и возмущающих воздействий.
\begin{table}[h!]
	\caption{Исходные данные}
	\renewcommand{\arraystretch}{1.8} %строки
	%\renewcommand{\tabcolsep}{1cm} %столбцы
	\begin{tabular}{|c|c|c|c|c|c|c|c|c|}
		\hline $W_0(s)$ & $W_1(s)$ & $g = A$ & $g = Vt$ & $g = at^2/2$ &  $f_1$ & $f_2$ & Сигнал задания\\
		\hline $\displaystyle{\frac{2}{0,5s^2+2s+1}}$& $\displaystyle{\frac{s+2}{0,5s^2+2s+1}}$ & 2 & t & $0.2t^2$ & -0.5 & -0.5 & $2+0.5t$\\
		\hline
	\end{tabular}	
\end{table}

\newpage

\begin{center}
\section{Исследование системы с астатизмом нулевого порядка}
\end{center}

\subsection{Исследование стационарного режима работы: $g(t)=A$} 
На рисунке 2 представлена структурная схема системы при входном воздействии \\$g=2$, представлены графики переходных процессов (рисунок 3) и переходные характеристики ошибок (рисунок 4) при различных значениях $k$. 
\begin{figure}[h!]
	\centering
	\includegraphics[width=1\linewidth]{scheme1.png}
	\caption{Структурная схема системы с астатизмом нулевого порядка}
\end{figure}
\begin{figure}[H]
	\centering
	\includegraphics[width=1\linewidth]{one.eps}
	\caption{Переходные характеристики системы для стационарного режима работы}
\end{figure}
\begin{figure}[H]
	\centering
	\includegraphics[width=1\linewidth]{one-2.eps}
	\caption{Переходные характеристики для ошибки}
\end{figure}
\par С помощью расчета проверим получившееся на графике значения установившейся ошибки:
\begin{equation}
	e = \frac{A}{(1+k)}
\end{equation}
при $k=1$: $\varepsilon = \displaystyle{\frac{A}{1+k}}={\frac{2}{2}}=0.5;$\\
при $k=5$: $\varepsilon = \displaystyle{\frac{1}{6}}=0,16;$\\
при $k=10$: $\varepsilon = \displaystyle{\frac{1}{11}}=0,09;$\\

\newpage
\subsection{Исследование режима движения с постоянной скоростью: \\$g(t)=Vt$} 
На рисунке 5 представлена переходная характеристика системы при входном воздействии $g=2t$.
\begin{figure}[H]
	\centering
	\includegraphics[width=1\linewidth]{one-two.eps}
	\caption{Переходные характеристики системы для движения с постоянной скоростью}
\end{figure}
\begin{figure}[H]
	\centering
	\includegraphics[width=1\linewidth]{one-two-2.eps}
	\caption{Переходные характеристики для ошибки движения с постоянной скоростью}
\end{figure}
Для статической системы при линейно нарастающем входном воздействии $g(t)=Vt$ имеем:
\begin{equation}
    \varepsilon = \lim_{s\to0} s\frac{1}{1+H(s)W(s)}G(s) = \infty.
\end{equation}\par
\newpage
\begin{center}
\section{Исследование системы с астатизмом первого порядка}
\end{center}\par
Структурная схема моделируемой системы представлена на рисунке 1, где $H(s) = \displaystyle{\frac{k}{s}},\\W(s)=\displaystyle{\frac{s+2}{0,5s^2+2s+1}}$.

\subsection{Исследование стационарного режима работы: $g(t)=A$} 
На рисунке 6 представлена структурная схема системы при входном воздействии \\$g=1$, представлены графики переходных процессов (рисунок 7) и переходные характеристики ошибок (рисунок 8) при различных значениях $k$. 
\begin{figure}[H]
	\centering
	\includegraphics[width=1\linewidth]{scheme2-1.png}
	\caption{Структурная схема системы с астатизмом моделируемой системы}
\end{figure}
Для статической системы при постоянном входном воздействии $g(t)=1$ имеем:
\begin{equation}
    \varepsilon = \lim_{s\to0} s\frac{1}{1+H(s)W(s)}G(s) = 0.
\end{equation}
\begin{figure}[H]
	\centering
	\includegraphics[width=1\linewidth]{two.eps}
	\caption{Переходные характеристики системы для стационарного режима работы}
\end{figure}
\begin{figure}[H]
	\centering
	\includegraphics[width=1\linewidth]{two-2.eps}
	\caption{Переходные характеристики для ошибки}
\end{figure}

\subsection{Исследование режима движения с постоянной скоростью: \\$g(t)=Vt$} 
На рисунке 9 представлена переходная характеристика системы при входном воздействии $g=2t$, на рисунке 10  - переходные характеристики для ошибки.
\begin{figure}[H]
	\centering
	\includegraphics[width=1\linewidth]{two-two-1.eps}
	\caption{Переходные характеристики системы для движения с постоянной скоростью}
\end{figure}
При линейно нарастающем воздействии $g(t)=Vt$ предельное значение установившейся ошибки будет равно:
\begin{equation}
    \varepsilon = \lim_{s\to 0}s\frac{1}{1+W(s)}*\frac{V}{s^2} = \lim_{s\to 0}\frac{s}{s+k}\frac{V}{s} = \frac{V}{k}.
\end{equation}
Тогда при $k=1$: $\varepsilon = \displaystyle{\frac{1}{2} = 0.5;}$\\
при $k=5$: $\varepsilon = \displaystyle{\frac{1}{6} = 0.16;}$\\
при $k=10$: $\varepsilon = \displaystyle{\frac{1}{11} = 0.09}$
\begin{figure}[H]
	\centering
	\includegraphics[width=1\linewidth]{two-two-2.eps}
	\caption{Переходные характеристики для ошибки}
\end{figure}

\subsection{Исследование режима движения с постоянным ускорением: \\$g(t)=at^2/2$} 
На рисунке 11 представлена переходная характеристика системы при входном воздействии $g=0.2t^2$ и ошибка на рисунке 12.
\begin{figure}[H]
	\centering
	\includegraphics[width=1\linewidth]{two-three-1.eps}
	\caption{Переходные характеристики системы для движения с постоянным ускорением}
\end{figure}
\begin{figure}[H]
	\centering
	\includegraphics[width=1\linewidth]{two-three-2.eps}
	\caption{Переходные характеристики для ошибки при входном воздействии $g=0.2t^2$}
\end{figure}
\newpage
\begin{center}
\section{Исследование влияний внешних возмущений}
\end{center}\par
Структурная схема возмущённой системы при входном воздействии $g=2+0.5t$ представлена на рисунке 13, также представлены графики переходных процессов (рисунок 14) и переходные характеристики ошибок (рисунок 15) при различных значениях $k$.
\begin{figure}[H]
    \centering
    \includegraphics[width=1\linewidth]{scheme-3.png}
    \caption{Структурная схема системы при влиянии внешних возмущений}
\end{figure}
\begin{figure}[H]
    \centering
    \includegraphics[width=1\linewidth]{three-one.eps}
    \caption{Переходные характеристики системы при влиянии внешних возмущений}
\end{figure}
\begin{figure}[H]
    \centering
    \includegraphics[width=1\linewidth]{three-two.eps}
    \caption{Переходные характеристики для ошибки}
\end{figure}

\newpage
\begin{center}
\section{Исследование установившейся ошибки при произвольном входном воздействии}
\end{center}\par
 Структурная схема представлена на рисунке 1, где $H(s) = 1, W(s) = \displaystyle{\frac{2}{0,5s^2 + 2s + 1}}$, а задающее воздействие $g(t) = {2 + 0.5t}$.
 В ходе моделирования заданной системы (рисунок 15) был получен график переходного процесса, представленный на рисунке 16. Из него видно, что предельное значение ошибки стремится к $\infty$. Схема моделирования системы представленна на рисунке 13.
\begin{figure}[H]
    \centering
    \includegraphics[width=1\linewidth]{scheme-4.png}
    \caption{Структурная схема системы при произвольном входном воздействии}
\end{figure}
\begin{figure}[H]
    \centering
    \includegraphics[width=1\linewidth]{four-one.eps}
    \caption{Переходной процесс в замкнутой системе при произвольном входном воздействии}
\end{figure}
Получим приближенное аналитическое выражение для установившейся ошибки слежения путём разложения в ряд Тейлора передаточную функцию замкнутой системы по ошибке слежения.
Передаточная функция замкнутой системы по ошибке слежения выглядит так:
\begin{equation}
   \Phi_e(s) = \frac{1}{1 + W(s)} = \frac{0.5s^2 + 2s + 1}{0.5s^2 + 2s + 3}.
\end{equation}\par
При произвольном входном воздействии выражение установившейся ошибки будет выглядеть следующим образом:
\begin{equation}
    e_y(t) = \Phi_e(s)|_{s=0}g(t) + \left.\frac{d\Phi_e(s)}{ds}\right|_{s=0}\dot{g}(t) + \left.\frac{d^2\Phi_e(s)}{ds^2}\right|_{s=0}\frac{\ddot{g}(t)}{2!}.
\end{equation}\par
Найдём производные $g(t)$ и $\Phi_e(s)$:
\begin{equation}
    g(t) = {2 + 0.5t} 
    \Phi_e(s)|_{s=0} = \frac{0.5s^2 + 2s + 1}{0.5s^2 + 2s + 3} = {1}{3} = 0.33 \\
    \dot{g}(t) = {1} 
    \Phi_e(s){ds}|_{s=0} = \frac{8s + 16}{(0.5s^2 + 4s + 6)^2} = {0.44} \\
    \ddot{g}(t) = {0} \\
\end{equation}
\par
Тогда получаем выражение ошибки $e_y(t)$:
\begin{equation}
e_y(t) = {0.33(2 + 0.5t) + 0.44 * 0.5 + 0} = 0.165t + 0.88.
\end{equation}
\par
Убедимся, что графики расчетной и экспериментально определённой установившейся ошибки слежения совпадают для этого построим их на одном графике, представленном на рисунке 17.
\begin{figure}[H]
    \centering
    \includegraphics[width=1\linewidth]{four-two.eps}
    \caption{Графики ошибок}
\end{figure}

\newpage
\begin{center}
\section{Вывод}
\end{center}
\par
В ходе лабораторной работы были исследованы системы с разным порядком астатизма, при разных условиях: при влиянии внешних возмущений и при произвольном и заданном входном воздействии. Были построены переходные характеристики для всех случаев и найдены значения установившихся ошибок. Данные исследования позволяют сделать вывод о том что, установившееся значение ошибки можно изменить путём увеличения или уменьшения общего коэффициента усиления разомкнутой системы, а также путём снижения или повышения порядка астатизма. 
\end{document}